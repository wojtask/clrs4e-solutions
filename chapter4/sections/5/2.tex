The recurrence describing the worst-case running time of the professor's algorithm is
\[
    T(n) = aT(n/4)+\Theta(n^2).
\]
For this algorithm to run asymptotically faster than Strassen's algorithm, it should be $T(n)=o(n^{\lg7})$.
Observe that neither case~2 nor 3 of the master theorem can't resolve $T(n)$ to be asymptotically less than the driving function $f(n)$.
Since $f(n)=\Theta(n^2)$, neither of those cases is interesting for us.

That leaves us with case~1, for which to apply there should be $\Theta(n^2)=O(n^{\log_4a-\epsilon})$ for some $\epsilon>0$, which holds for $a>16$.
Then the solution to the recurrence is $T(n)=\Theta(n^{\log_4a})$, so
\begin{align*}
    \log_4a &= \frac{\lg a}{\lg4} \\[1mm]
    &= \frac{\lg a}{2} \\
    &= \lg\sqrt{a} \\
    &< \lg7,
\end{align*}
holds when $a<49$.

The largest integer $a$ we are looking for, is $a=48$.
