Let $\phi\ge1$ be arbitrary, and pick $d=\phi^2+1$.
Let $1\le\psi\le\phi$, so that $1^2\le\psi^2\le\phi^2$.
For $n\ge0$ we have
\begin{align*}
    f(n)/d &= n^2/(\phi^2+1) \\
    &\le n^2 \\
    &\le \psi^2n^2 && \text{($=f(\psi n)$)} \\
    &\le \phi^2n^2 \\
    &\le (\phi^2+1)n^2 \\
    &= df(n),
\end{align*}
which proves that $f(n)=n^2$ satisfies the polynomial-growth condition.

Now for $f(n)=2^n$ pick $\phi=\psi=2$ and let $d>1$ be any constant.
Then
\begin{align*}
    f(\psi n) &= 2^{\psi n} \\
    &= 2^{2n} \\
    &= 2^n\cdot2^n \\
    &\le d2^n \\
    &= df(n),
\end{align*}
but we must have $d\ge2^n$, for the second to last step to hold for all sufficiently large $n$, which is a contradiction.
We conclude that $f(n)=2^n$ doesn't satisfy the polynomial-growth condition.
