In this exercise, when calculating integrals we ignore the constant of integration, since in all cases it is swallowed by the $\Theta$-notation in equation (4.23).
\vspace*{2\baselineskip}

\subexercise
This is an Akra-Bazzi recurrence with $a_1=a_2=a_3=1$, $b_1=2$, $b_2=3$, $b_3=6$, and $f(n)=n\lg n$.
Note that $1/2+1/3+1/6=1$, so $p=1$.
We'll also need the following integral:
\begin{align*}
    I &= \int\frac{\lg x}{x}\,dx \\
    &= \int\lg x(\ln x)'\,dx \\
    &= \lg x\ln x-\int\frac{\ln x}{x\ln2}\,dx && \text{(integrating by parts)} \\
    &= \frac{\lg^2x}{\lg e}-\int\frac{\lg x}{x}\,dx && \text{(by equation (3.19))} \\[1mm]
    &= \frac{\lg^2x}{\lg e}-I,
\end{align*}
which gives
\[
    I = \frac{\lg^2x}{2\lg e}.
\]
Finally, by identity (4.23) we have
\begin{align*}
    T(n) &= \Theta\left(n\left(1+\int_1^n\frac{x\lg x}{x^2}\,dx\right)\right) \\
    &= \Theta\left(n\left(1+\int_1^n\frac{\lg x}{x}\,dx\right)\right) \\
    &= \Theta\biggl(n\,\biggl(1+\biggl[\frac{\lg^2x}{2\lg e}\biggr]_1^n\biggr)\biggr) \\
    &= \Theta\biggl(n\,\biggl(1+\frac{\lg^2n}{2\lg e}\biggr)\biggr) \\
    &= \Theta(n\lg^2n).
\end{align*}

\subexercise
Here we have $a_1=3$, $a_2=8$, $b_1=3$, $b_2=4$, and $f(n)=n^2/\lg n$.
We need to find $p$ such that
\[
    \frac{3}{3^p}+\frac{8}{4^p} = 1.
\]
Observe that $3/3^1+8/4^1=1+2>1$ and $3/3^2+8/4^2=1/3+1/2<1$, thus it has to be $1<p<2$.

If we used the Akra-Bazzi method directly to solve $T(n)$, we would need to determine
\[
    I = \int_1^n\frac{x^{1-p}}{\lg x}\,dx.
\]
However, the function being integrated is undefined for $x=1$, making $I$ an improper integral, and as it turns out,
\[
    \lim_{t\to1^+}\int_t^n\frac{x^{1-p}}{\lg x}\,dx = \infty,
\]
making $I$ divergent.
As such, $I$ can't be used in equation (4.23) to determine the solution to the recurrence.

Instead, we will calculate a (finite) lower bound on $I$ and will use it in equation (4.23) to find an asymptotic lower bound on $T(n)$.
Since $\lg x\le\lg n$ in the entire interval of integration, we have
\begin{align*}
    I &\ge \int_1^n\frac{x^{1-p}}{\lg n}\,dx \\[1mm]
    &= \frac{1}{\lg n}\int_1^nx^{1-p}\,dx \\[1mm]
    &= \frac{1}{\lg n}\left[\frac{x^{2-p}}{2-p}\right]_1^n \\[1mm]
    &= \frac{(n^{2-p}-1)}{(2-p)\lg n}.
\end{align*}
Then,
\begin{align*}
    T(n) &= \Theta(n^p(1+I)) \\
    &= \Omega\left(n^p\left(1+\frac{(n^{2-p}-1)}{(2-p)\lg n}\right)\right) \\
    &= \Omega\left(n^p+\frac{(n^2-n^p)}{(2-p)\lg n}\right) \\
    &= \Omega(n^2/\lg n) && \text{(since $n^p=o(n^2/\lg n)$)}.
\end{align*}

Hinted by the asymptotic lower bound we've just obtained, we'll prove that this bound is tight, using the substitution method.
Assume by induction that
\[
    T(m) \le cm^2/\lg m
\]
for all $n_0\le m<n$, where $c$, $n_0>0$ are constants that we'll determine later.
In particular, if $n\ge4n_0$, then the induction hypothesis holds for $n/3$ and $n/4$, so substituting it into the formula for $T(n)$ yields
\begin{align*}
    T(n) &= 3T(n/3)+8T(n/4)+\frac{n^2}{\lg n} \\
    &\le \frac{3c(n/3)^2}{\lg(n/3)}+\frac{8c(n/4)^2}{\lg(n/4)}+\frac{n^2}{\lg n} \\
    &= \frac{cn^2}{3(\lg n-\lg3)}+\frac{cn^2}{2(\lg n-2)}+\frac{n^2}{\lg n}.
\end{align*}
Note that
\[
    3(\lg n-\lg3) \ge (8/3)\lg n
\]
as long as $n\ge2^{3\cdot3\lg3}=3^9=19{,}683$, and
\[
    2(\lg n-2) \ge (5/3)\lg n
\]
as long as $n\ge2^{3\cdot2\cdot2}=2^{12}=4{,}096$.

Thus, if we let $n_0\ge\lceil3^9\!/4\rceil$, so that $n\ge3^9$, we have
\begin{align*}
    T(n) &\le \frac{cn^2}{(8/3)\lg n}+\frac{cn^2}{(5/3)\lg n}+\frac{n^2}{\lg n} \\
    &= ((3/8)c+(3/5)c+1)\cdot\frac{n^2}{\lg n} \\
    &= ((39/40)c+1)\cdot\frac{n^2}{\lg n} \\
    &\le \frac{cn^2}{\lg n},
\end{align*}
where the last step holds if $39c/40+1\le c$, or $c\ge40$.

Now we need to prove that the inductive hypothesis holds for the base cases, that is $T(n)\le cn^2/\lg n$ when $n_0\le n<4n_0$.
Pick $n_0=\lceil3^9\!/4\rceil$, according to the earlier constraint.
By our convention, $T(n)$ is algorithmic, so we can view all $T(n)$, where $n<3^9$, as constants.
Picking $c=\max{T(\lceil3^9\!/4\rceil),T(\lceil3^9\!/4\rceil+1),\dots,T(3^9-1)}$ yields $T(n)\le c<cn^2/\lg n$ for all $\lceil3^9\!/4\rceil\le n<3^9$, which completes the proof that $T(n)=O(n^2/\lg n)$.

Combining this result with the asymptotic lower bound, we have that $T(n)=\Theta(n^2/\lg n)$.

\subexercise
In this recurrence we have $a_1=2/3$, $a_2=1/3$, $b_1=3$, $b_2=3/2$, and $f(n)=\lg n$.
Observe that $a_1+a_2=1$, in which case we immediately obtain $p=0$.
Then,
\begin{align*}
    T(n) &= \Theta\left(n^0\left(1+\int_1^n\frac{\lg x}{x}\,dx\right)\right) \\
    &= \Theta\biggl(1+\biggl[\frac{\lg^2x}{2\lg e}\biggr]_1^n\biggr) && \text{(we calculated this integral in part (a))} \\
    &= \Theta\biggl(1+\frac{\lg^2n}{2\lg e}\biggr) \\
    &= \Theta(\lg^2n).
\end{align*}

\subexercise
Here, $a_1=1/3$, $b_1=3$, and $f(n)=1/n$.
The term $\frac{1/3}{3^p}=3^{-p-1}$ equals 1, if $p=-1$.
We know from calculus that $\int dx/x=\ln x$, and thus, by equation (4.23):
\begin{align*}
    T(n) &= \Theta\left(n^{-1}\left(1+\int_1^n\frac{1/x}{x^0}\,dx\right)\right) \\
    &= \Theta\left((1/n)\left(1+\int_1^n\frac{dx}{x}\right)\right) \\
    &= \Theta\bigl((1/n)\bigl(1+[\ln x]_1^n\bigr)\bigr) \\
    &= \Theta\bigl((1/n)(1+\ln n)\bigr) \\
    &= \Theta((\lg n)/n).
\end{align*}

\subexercise
Here, $a_1=a_2=3$, $b_1=3$, $b_2=3/2$, and $f(n)=n^2$.
We evaluate the sum
\[
    \frac{3}{3^p}+\frac{3}{(3/2)^p} = \frac{1+2^p}{3^{p-1}}
\]
using different values for $p$, to find that the sum achieves 1 when $p=3$.
Since $\int dx/x^2=-1/x$, we have
\begin{align*}
    T(n) &= \Theta\left(n^3\left(1+\int_1^n\frac{x^2}{x^4}\,dx\right)\right) \\
    &= \Theta\left(n^3\left(1+\int_1^n\frac{dx}{x^2}\right)\right) \\
    &= \Theta\bigl(n^3\bigl(1+[-1/x]_1^n\bigr)\bigr) \\
    &= \Theta\bigl(n^3\bigl(1+(-1/n+1)\bigr)\bigr) \\
    &= \Theta(n^3(2-1/n)) \\
    &= \Theta(n^3).
\end{align*}
