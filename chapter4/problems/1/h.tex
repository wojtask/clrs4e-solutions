This is not a master recurrence, so we can't apply the master method to it.
Instead, we'll guess the solution and verify it using the substitution method.

The recurrence adds up the square of the problem size to the total cost, and decreases the problem size by 2.
The partial costs are therefore $n^2$, $(n-2)^2$, $(n-4)^2$, $\dots$, so the total cost resolves roughly to
\begin{align*}
    \sum_{k=0}^n(n-k)^2 &= \sum_{k=0}^nk^2 \\[1mm]
    &= \frac{n(n+1)(2n+1)}{6} && \text{(by equation (A.4))} \\
    &= \Theta(n^3).
\end{align*}
Let's then adopt an inductive hypothesis that $T(n)\ge c_1n^3$ for all $n\ge n_1$, where $c_1$, $n_1>0$ are constants.
Assume by induction that the hypothesis holds for all numbers at least as big as $n_1$ and less than $n$.
Let $n\ge n_1+2$, so that $T(n-2)\ge c_1(n-2)^3$.
By substitution,
\begin{align*}
    T(n) &\ge c_1(n-2)^3+n^2 \\
    &= c_1(n^3-3\cdot2n^2+3\cdot2^2n-2^3)+n^2 \\
    &= c_1n^3+(1-6c_1)n^2+12c_1n-8c_1.
\end{align*}
Let $f_c(n)=(1-6c)n^2+12cn-8c$, where $c$ is a positive constant.
We have
\begin{align*}
    f_c(n) &> (1-6c)n^2-8c \\
    &\ge (1-6c)n^2-8cn^2 && \text{(as long as $n\ge1$)} \\
    &= (1-14c)n^2 \\
    &\ge 0 && \text{(as long as $c\le1/14$)}.
\end{align*}
Thus, we get that for $c_1\le1/14$, $T(n)\ge c_1n^3+f_{c_1}(n)\ge c_1n^3$ for all $n\ge n_1+2$, which proves the inductive hypothesis holds for the inductive case.

Now let $n_1\le n<n_1+2$ and let's pick $n_1=1$.
Both $T(1)$ and $T(2)$ are constants, and picking $c_1=\min{T(1),T(2)/8}$ satisfies both inequalities $T(1)\ge c_1\cdot1^3$ and $T(2)\ge c_1\cdot2^3$, handling the base cases.

To show the upper bound on $T(n)$ we follow a similar reasoning as above, adopting the inductive hypothesis $T(n)\le c_2n^3$ for all $n\ge n_2$, where $c_2$, $n_2>0$ are constants.
At some point we'll have that $T(n)\le c_2n^3+f_{c_2}(n)$.
Since
\begin{align*}
    f_c(n) &< (1-6c)n^2+12cn \\
    &\le (1-6c)n^2+4cn^2 && \text{(as long as $n\ge3$)} \\
    &= (1-2c)n^2 \\
    &\le 0 && \text{(as long as $c\ge1/2$)},
\end{align*}
we conclude that when $c_2\ge1/2$ and $n_2\ge1$, it holds that $T(n)\le c_2n^3$ for all $n\ge n_2+2$.

Let's choose $n_2=1$.
We handle both $T(1)\le c_2\cdot1^3$ and $T(2)\le c_2\cdot2^3$ by choosing $c_2=\max{T(1),T(2)/8}$, which finishes the inductive proof of the upper bound on $T(n)$.

We've proven that $c_1n^3\le T(n)\le c_2n^3$ for some constants $c_1$, $c_2>0$ and for all $n\ge1$, and thus $T(n)=\Theta(n^3)$.
