Let $S$ be the sample space containing all possible sequences of outcomes during $n$ Bernoulli trials.
A set $A$ corresponds to a sequence $s\in S$, and $X(s)$ is the number of successes in $s$.

We'll show how to obtain a set $A'$ of Bernoulli trials by performing experiments on the trials of $A$.
Let $A_i$ denote the event that the $i$th trial of $A$ is a success, and let $A_i'$ be the event that the $i$th trial of $A'$ is a success.
If $A_i$ occurs, we'll make $A_i'$ to also occur.
Otherwise, a success in the $i$th trial of $A'$ will be generated with some probability $r_i$.
Then,
\begin{align*}
    p_i' &= \Pr{A_i'} \\
    &= \Pr{A_i'\cap A_i}+\Pr{A_i'\cap\overline{A_i}} \\
    &= p_i+(1-p_i)\cdot r_i,
\end{align*}
and so,
\[
    r_i = \frac{p_i'-p_i}{1-p_i}.
\]

Reusing the sample space $S$, we'll denote by $X'(s)$ the number of successes in a series of the $n$ trials obtained by following the above procedure, based on the outcomes in $s$.
For any $s\in S$ and its corresponding set $A$, it's clear that after constructing the set $A'$ this way, it's impossible that there are fewer successes in $A'$ than in the original set $A$.
In other words, $X'(s)\ge X(s)$ holds.
Using \refExercise{C.3-7}, we get the desired result.
