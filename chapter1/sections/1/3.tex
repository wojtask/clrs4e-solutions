We'll discuss the properties of \concept{arrays}\dash one of the simplest non-trivial data structures widely used in computer programming.
The strengths of arrays include:
\begin{description}
    \item[Simplicity:] Arrays are a straightforward and easy-to-understand data structure.
    They have a simple interface and operations like insertion and deletion are well-defined.
    This simplicity makes arrays a popular choice for beginners and for scenarios where the data size is known in advance.
    \item[Random Access:] Arrays provide constant-time access to individual elements.
    Each element in an array can be accessed directly by its index, allowing for efficient random access operations.
    This makes arrays suitable for scenarios where quick access to elements is required.
    \item[Efficiency:] Arrays have a fixed size and provide efficient memory allocation.
    The elements in an array are stored contiguously in memory, allowing for efficient traversal and manipulation of the array.
    This makes arrays well-suited for operations that involve iterating over elements or performing arithmetic computations.
\end{description}

Arrays also have some limitations that may make them less useful or efficient in certain problems.
These are:
\begin{description}
    \item[Fixed Size:] Arrays have a fixed size, meaning they cannot be dynamically resized once created.
    This can be a limitation when the number of elements in the array needs to change dynamically.
    Resizing an array typically involves creating a new, larger array and copying the existing elements, which can be inefficient.
    \item[Wasted Space:] Arrays require contiguous memory allocation, which can lead to wasted space when the array is not fully occupied.
    If an array is created with a larger capacity than needed, the unused space remains allocated, resulting in wasted memory.
    This is especially problematic when dealing with large arrays or in situations where memory is a scarce resource.
    \item[Insertion and Deletion:] Inserting or deleting elements in an array can be inefficient.
    If an element is inserted or deleted in the middle of the array, all the subsequent elements need to be shifted to accommodate the change.
    This operation takes the time proportional to the number of elements in the array.
    Therefore, arrays are not the best choice when frequent insertions or deletions at arbitrary positions are required.
    \item[Homogeneousness:] Arrays have a fixed data type, which means they can only store elements of a single type.
    This can be limiting when working with heterogeneous data.
\end{description}
