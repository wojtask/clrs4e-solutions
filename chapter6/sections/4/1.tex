See \refFigure{6.4-1}.
\begin{figure}[htb]
    \hspace*{-1.5cm}
    \newlength{\figurewidth}\setlength{\figurewidth}{\textwidth+1.5cm}
    \begin{minipage}{\figurewidth}
        \subcaptionbox{\label{fig:6.4-1a}}[0.33\figurewidth]{\subimport{figures/1/}{a.tikz}}
        \subcaptionbox{\label{fig:6.4-1b}}[0.33\figurewidth]{\subimport{figures/1/}{b.tikz}}
        \subcaptionbox{\label{fig:6.4-1c}}[0.33\figurewidth]{\subimport{figures/1/}{c.tikz}}
        \par\vspace{1.5\vertexsize}
        \subcaptionbox{\label{fig:6.4-1d}}[0.33\figurewidth]{\subimport{figures/1/}{d.tikz}}
        \subcaptionbox{\label{fig:6.4-1e}}[0.33\figurewidth]{\subimport{figures/1/}{e.tikz}}
        \subcaptionbox{\label{fig:6.4-1f}}[0.33\figurewidth]{\subimport{figures/1/}{f.tikz}}
        \par\vspace{1.5\vertexsize}
        \subcaptionbox{\label{fig:6.4-1g}}[0.33\figurewidth]{\subimport{figures/1/}{g.tikz}}
        \subcaptionbox{\label{fig:6.4-1h}}[0.33\figurewidth]{\subimport{figures/1/}{h.tikz}}
        \subcaptionbox{\label{fig:6.4-1i}}[0.33\figurewidth]{\subimport{figures/1/}{i.tikz}}
        \par\vspace{1.5\vertexsize}
        \subcaptionbox{\label{fig:6.4-1j}}[0.33\figurewidth]{\subimport{figures/1/}{j.tikz}}
    \end{minipage}
    \caption{The operation of \proc{Heapsort} on the array $A=\langle$5, 13, 2, 25, 7, 17, 20, 8, 4$\rangle$.\,
    \textbf{(a)}\, The max-heap built by \proc{Build-Max-Heap} in line 1.\,
    \textbf{(b)--(i)}\, The max-heap and the nodes removed from it after each call to \proc{Max-Heapify} in line 5.\,
    \textbf{(j)}\, The resulting sorted array $A$.} \label{fig:6.4-1}
\end{figure}
