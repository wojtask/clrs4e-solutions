See \refFigure{6.3-1}.
\begin{figure}[htb]
    \captionsetup[subfigure]{}
    \subcaptionbox{\label{fig:6.3-1a}}[0.5\textwidth]{\begin{tikzpicture}[
    container/.style = {draw=none, fill=none, node distance=1mm, every node/.style=tree node},
    level/.append style = {sibling distance=30mm/2^#1}
]
    \node[container] (tree 1) {
        \begin{tikzpicture}[anchor=center]
            \node {$x$}
            child {node {$y$}
            child {node {$z$}}
            };
        \end{tikzpicture}
    };
    \node[container, right=of tree 1] (tree 2) {
        \begin{tikzpicture}[anchor=center]
            \node {$x$}
            child {node {$z$}
            child {node {$y$}}
            };
        \end{tikzpicture}
    };
    \node[container, right=of tree 2] (tree 3) {
        \begin{tikzpicture}[anchor=center]
            \node {$y$}
            child {node {$x$}
            child {node {$z$}}
            };
        \end{tikzpicture}
    };
\end{tikzpicture}
}
    \subcaptionbox{\label{fig:6.3-1b}}[0.5\textwidth]{\tikzset{
    every matrix/.style = {indexed array},
    moved cell/.style = {row 2 column #1/.style={nodes={fill=lightblue}}},
    shifted cell/.style = {row 2 column #1/.style={nodes={fill=tan}}},
}

\begin{tikzpicture}
    \node[label=1] {5}
        child {node[label=2] {3}
            child {node[label=4] {22}
                child {node[label=8] {10}}
                child {node[label=9] {9}}
            }
            child {node[label=5] {84}}
        }
        child {node[active, label=3, label=left:{$i$}] {17}
            child {node[label=6] {19}}
            child {node[label=7] {6}
                child[missing]
                child {node[draw=none, fill=none] {} edge from parent[draw=none]}
            }
        };
\end{tikzpicture}
}
    \par\vspace{7mm}
    \subcaptionbox{\label{fig:6.3-1c}}[0.5\textwidth]{\tikzset{
    every matrix/.style = {indexed array},
    moved cell/.style = {row 2 column #1/.style={nodes={fill=lightblue}}},
    shifted cell/.style = {row 2 column #1/.style={nodes={fill=tan}}},
}

\begin{tikzpicture}
    \node {17}
        child {node {13}
            child {node {8}
                child {node[discarded, label=left:{$i$}] {20} edge from parent[draw=none]}
                child {node[discarded] {25} edge from parent[draw=none]}
            }
            child {node {7}}
        }
        child {node {5}
            child {node {4}}
            child {node {2}
                child[missing]
                child {node[draw=none, fill=none, label=right:{}] {} edge from parent[draw=none]}
            }
        };
\end{tikzpicture}
}
    \subcaptionbox{\label{fig:6.3-1d}}[0.5\textwidth]{\tikzset{
    every matrix/.style = {indexed array},
    moved cell/.style = {row 2 column #1/.style={nodes={fill=lightblue}}},
    shifted cell/.style = {row 2 column #1/.style={nodes={fill=tan}}},
}

\begin{tikzpicture}
    \node {13}
        child {node {8}
            child {node {2}
                child {node[discarded] {20} edge from parent[draw=none]}
                child {node[discarded] {25} edge from parent[draw=none]}
            }
            child {node {7}}
        }
        child {node {5}
            child {node {4}}
            child {node[discarded, label=left:{$i$}] {17} edge from parent[draw=none]
                child[missing]
                child {node[draw=none, fill=none] {} edge from parent[draw=none]}
            }
        };
\end{tikzpicture}
}
    \par\vspace{7mm}
    \subcaptionbox{\label{fig:6.3-1e}}[0.5\textwidth]{\tikzset{
    every matrix/.style = {indexed array},
    moved cell/.style = {row 2 column #1/.style={nodes={fill=lightblue}}},
    shifted cell/.style = {row 2 column #1/.style={nodes={fill=tan}}},
}

\begin{tikzpicture}
    \node[label=1] {84}
        child {node[label=2] {22}
            child {node[label=4] {10}
                child {node[label=8] {5}}
                child {node[label=9] {9}}
            }
            child {node[label=5] {3}}
        }
        child {node[label=3] {19}
            child {node[label=6] {17}}
            child {node[label=7] {6}
                child[missing]
                child {node[draw=none, fill=none] {} edge from parent[draw=none]}
            }
        };
\end{tikzpicture}
}
    \caption{The operation of \proc{Build-Max-Heap} on the array $A=\langle$5, 3, 17, 10, 84, 19, 6, 22, 9$\rangle$.\,
    \textbf{(a)}\, The array $A$ and the binary tree it represents before the first call to \proc{Max-Heapify} in line 3.\,
    \textbf{(b)--(d)}\, The data structure before each subsequent call to \proc{Max-Heapify}.\,
    \textbf{(e)}\, The resulting max-heap after \proc{Build-Max-Heap} finishes.} \label{fig:6.3-1}
\end{figure}
