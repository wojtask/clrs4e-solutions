See \refFigure{6.3-1}.
\begin{figure}[htb]
    \subcaptionbox{\label{fig:6.3-1a}}[0.5\textwidth]{\subimport{figures/1/}{a.tikz}}
    \subcaptionbox{\label{fig:6.3-1b}}[0.5\textwidth]{\subimport{figures/1/}{b.tikz}}
    \par\vspace{1.5\vertexsize}
    \subcaptionbox{\label{fig:6.3-1c}}[0.5\textwidth]{\subimport{figures/1/}{c.tikz}}
    \subcaptionbox{\label{fig:6.3-1d}}[0.5\textwidth]{\subimport{figures/1/}{d.tikz}}
    \par\vspace{1.5\vertexsize}
    \subcaptionbox{\label{fig:6.3-1e}}[0.5\textwidth]{\subimport{figures/1/}{e.tikz}}
    \caption{The operation of \proc{Build-Max-Heap} on the array $A=\langle$5, 3, 17, 10, 84, 19, 6, 22, 9$\rangle$.\,
    \textbf{(a)}\,~The array $A$ and the binary tree it represents before the first call to \proc{Max-Heapify} in line 3.\,
    \textbf{(b)--(d)}\,~The data structure before each subsequent call to \proc{Max-Heapify}.\,
    \textbf{(e)}\,~The resulting max-heap after \proc{Build-Max-Heap} finishes.} \label{fig:6.3-1}
\end{figure}
