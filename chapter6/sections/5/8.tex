The procedure below shows the modification inspired by the inner loop of \proc{Insertion-Sort}.
Here, the \kw{while} loop moves some of the nodes on the path from $x$ toward the root down by 1 level to find a proper position where it can place $x$.
That proper position is determined by comparing the key of the next node on the path with the new key of $x$.

\begin{codebox}
\Procname{$\proc{Max-Heap-Increase-Key}'(A,x,k)$}
\li \If $k<\attrib{x}{key}$
\li     \Then \Error ``new key is smaller than current key''
        \End
\li $\attrib{x}{key}\gets k$
\li find the index $i$ in array $A$ where object $x$ occurs
\li \While $i>1$ and $\attrib{A[\proc{Parent}(i)]}{key}<k$
\li     \Do $A[i]\gets A[\proc{Parent}(i)]$
\li         map $A[i]$ to index $i$ in the array
\li         $i\gets\proc{Parent}(i)$
        \End
\li $A[i]\gets x$
\li map $x$ to index $i$ in the array
\end{codebox}
