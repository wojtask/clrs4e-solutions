The idea behind the \proc{Young-Search} algorithm is to keep shrinking the submatrix where the searched number might potentially be stored, depending on how it relates to the top right entry in that submatrix.

The algorithm will find out if a number $k$ is stored in an $m\times n$ Young tableau $Y$.
In the beginning, the submatrix to search through is the whole $Y[1\subarr m,1\subarr n]$.
If $k$ equals $Y[1,n]$, then of course the entry of value $k$ has been found in $Y$, so we can terminate.
Now suppose that $k>Y[1,n]$.
The element $Y[1,n]$ is the largest in row~1, so it is clear that $k$ is nowhere in row~1.
We can therefore eliminate row~1 from further consideration and keep looking for $k$ in the submatrix $Y[2\subarr m,1\subarr n]$.
Similarly, if $k<Y[1,n]$, then $k$ is smaller than the smallest element in column $n$.
Thus, $k$ can't be in column $n$, so let's resume searching in the submatrix $Y[1\subarr m,1\subarr n-1]$.
Once the submatrix becomes empty, it is clear that $k$ is nowhere in $Y$.

\begin{codebox}
\Procname{$\proc{Young-Search}(Y,m,n,k)$}
\li $i\gets1$
\li $j\gets n$
\li \While $i\le m$ and $j\ge1$
\li     \Do \If $k\isequal Y[i,j]$
\li             \Then \Return \const{true}
                \End
\li         \If $k>Y[i,j]$
\li             \Then $i\gets i+1$
\li             \Else $j\gets j-1$
                \End
        \End
\li \Return \const{false}
\end{codebox}

In each iteration of the \kw{while} loop, the algorithm decrements the number of columns or the number of rows in the submatrix left to be searched through.
The \kw{while} loop will therefore make at most $m+n$ iterations.
Since each iteration takes constant time, the total running time of the algorithm is $O(m+n)$.
