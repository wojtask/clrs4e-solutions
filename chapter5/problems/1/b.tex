Let $X_j$ for $j=1$, 2, \dots, $n$, and $X$ be the random variables defined in the previous part.
For $n_i=100i$ the value represented by the counter will increase by $n_{i+1}-n_i=100$ after a call to \proc{Increment}, with the probability of $1/(n_{i+1}-n_i)=1/100$.
For all $j=1$, 2, \dots, $n$ we get
\begin{align*}
    \Var{X_j} &= \E{X_j^2}-\Esquared{X_j} && \text{(by equation (C.31))} \\
    &= 0^2\cdot\left(1-\frac{1}{100}\right)+100^2\cdot\frac{1}{100}-1^2 \\
    &= 99,
\end{align*}
and since $X_1$, $X_2$, \dots, $X_n$ are pairwise independent, we obtain
\begin{align*}
    \Var{X} &= \Var{\sum_{j=1}^nX_j} \\
    &= \sum_{j=1}^n\Var{X_j} && \text{(by equation (C.33))} \\
    &= \sum_{j=1}^n99 \\
    &= 99n.
\end{align*}
