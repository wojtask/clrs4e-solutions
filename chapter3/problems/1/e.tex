The proof is symmetric to the one from part (d).
Let $b_1$ be a positive constant and let $m=(b_1/c_1)^{1/(d-k)}$.
Then, since $k<d$, the inequality $b_1n^k<c_1n^d$ holds for all $n>m$.
Thus, for any $b_1>0$,
\[
    0 \le b_1n^k < c_1n^d \le p(n)
\]
for all $n\ge\max{n_0,m+1}$.
Hence, $p(n)=\omega(n^k)$.
