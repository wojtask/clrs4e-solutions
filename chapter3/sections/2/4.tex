By the definition of $\Theta$-notation, $f(n)=\Theta(g(n))$ whenever there exist positive constants $c_1$, $c_2$, and $n_0$ such that
\[
    0 \le c_1g(n) \le f(n) \le c_2g(n)
\]
for all $n\ge n_0$.
This condition can be decomposed into the combination of inequalities $0\le c_1g(n)\le f(n)$ and $0\le f(n)\le c_2g(n)$, both holding for all $n\ge n_0$.
From the former follows $f(n)=\Omega(g(n))$ and from the latter follows $f(n)=O(g(n))$.

For the proof of the opposite direction, suppose that $f(n)=\Omega(g(n))$ and $f(n)=O(g(n))$.
The former means that there exist positive constants $c_1$ and $n_1$ such that $0\le c_1g(n)\le f(n)$ for all $n\ge n_1$, and the latter means that there exist positive constants $c_2$ and $n_2$ such that $0\le f(n)\le c_2g(n)$ for all $n\ge n_2$.
Then, by letting $n_0=\max{n_1,n_2}$ and merging both conditions, we get that
\[
    0 \le c_1g(n) \le f(n) \le c_2g(n)
\]
for all $n\ge n_0$.
Thus, $f(n)=\Theta(g(n))$.
