Let $f(n)=o(n)$.
Then, in particular, there exists a positive constant $n_0$ such that $0\le f(n)<n/2$ for all $n\ge n_0$.
If $k\ge0$, the function $n^k$ is monotonically increasing, so we have
\[
    (n+f(n))^k \ge n^k
\]
and
\begin{align*}
    (n+f(n))^k &\le (n+n/2)^k \\
    &= (3/2)^kn^k.
\end{align*}
We can satisfy the condition in the definition of $\Theta$-notation for all $n\ge n_0$, by letting $c_1=1$ and $c_2=(3/2)^k$.
In case where $k<0$, the function $n^k$ is monotonically decreasing, and the above inequalities need to be reversed:
\[
    (3/2)^kn^k \le (n+f(n)) \le n^k.
\]
Then it suffices to plug $c_1=(3/2)^k$ and $c_2=1$ to make the condition true.
Hence, $(n+o(n))^k=\Theta(n^k)$.

Similarly, we can show that $(n-o(n))^k=\Theta(n^k)$.
Suppose that $k\ge0$.
For any $f(n)=o(n)$ satisfying $0\le f(n)<n/2$ for all $n\ge n_0$, where $n_0$ is a positive constant, we have
\begin{align*}
    (n-f(n))^k &\ge (n-n/2)^k \\
    &= (1/2)^kn^k
\end{align*}
and
\[
    (n-f(n))^k \le n^k.
\]
In the definition of $\Theta$-notation we let $c_1=(1/2)^k$ and $c_2=1$, so that the condition is met for all $n\ge n_0$.
The case where $k<0$ is symmetric.

From inequalities (3.2), if $k\ge0$, then
\begin{gather*}
    (n-1)^k < \lceil n\rceil^k < (n+1)^k, \\
    (n-1)^k < \lfloor n\rfloor^k < (n+1)^k,
\end{gather*}
and symmetrically, if $k<0$.
Then, by the fact shown above, we have $\lceil n\rceil^k=\Theta(n^k)$ and $\lfloor n\rfloor^k=\Theta(n^k)$.
